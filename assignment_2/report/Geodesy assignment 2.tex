\documentclass[english]{report}
\usepackage[a4paper]{geometry}
\geometry{textwidth=0.8\paperwidth, textheight=0.9\paperheight, noheadfoot, nomarginpar}
\setlength{\topskip}{0mm}
\setlength{\parindent}{10mm}
\usepackage{geometry}
\geometry{verbose,headsep=50pt}
\usepackage{babel}
\usepackage{natbib}
\bibliographystyle{plainnat}
%\setcitestyle{authoryear,open={(},close={)}}
\usepackage{fancyhdr}
\usepackage{graphicx}
\usepackage{wrapfig}
\usepackage{gensymb}
\usepackage{booktabs}
\usepackage{amssymb}
\usepackage{amsmath}

\usepackage{caption}
\graphicspath{ {../images/} }
\usepackage{titlesec}
 \newcommand{\HRule}{\rule{\linewidth}{0.5mm}}

\titleformat{\chapter}{\normalfont\huge}{\thechapter.}{24pt}{\huge\it}

%removes chapternumber%
\usepackage{setspace}
%\doublespacing
\pagestyle{fancy}
\usepackage[colorlinks=false,hidelinks]{hyperref}
\begin{document}

\begin{titlepage} \begin{center} 
		% Upper part of the page. The '~' is needed because \\ % only works if a paragraph has started. 
\includegraphics[width=0.3\textwidth]{./logo}~\\[1cm] \textsc{\LARGE University of Cape Town}\\[1.5cm] \textsc{\Large APG4001S Geodesy}\\[0.5cm] 
% Title 
\HRule \\[0.4cm] { \huge \bfseries Determination of Normal Correction to Levelled Heights \\[0.4cm] } \HRule \\[1.5cm]
% Author and supervisor
  \noindent \begin{minipage}[t]{0.4\textwidth} \begin{flushleft} \large \emph{Author:}\\ Craig \textsc{Ferguson} \end{flushleft} \end{minipage}
% 
  \begin{minipage}[t]{0.4\textwidth} \begin{flushright} \large \emph{Supervisor:} \\ Dr.~R \textsc{Govind} \end{flushright} \end{minipage} \vfill 
% Bottom of the page 
  {\large \today} \end{center} \end{titlepage}
%\tableofcontents
%\listoffigures
\chapter*{The Normal Heights Correction}
\subsection*{Introduction}
When precise leveling along a route, variations gravity can cause misclosures in levelling loops. The corrections for these level height differences when  using a Normal Heighting system is discussed.A Normal Heighting System is with reference to above the quasi-geoid.
\subsection*{Formula}
\begin{equation}
			H^{N}_{B} -  H^{N}_{B} =\sum_{A}^{B}\delta n + \sum_{A}^{B}\dfrac{g-G}{G}.\delta n +H^{N}_{B}\dfrac{G-\bar{\gamma_{B}}}{G} -
			H^{N}_{A}\dfrac{G-\bar{\gamma_{A}}}{G} 
\end{equation}

The first term on the right part of the equation is the usual sum of height differences for levelling. The last three are the correction that is applied. 
The last two are of course required as the left hand side of the equation is a difference. The correction for each point is easy to identify using subscripts.

\subsection*{Gamma Specifics}
$ {\gamma} $ is the normal gravity on the ellipsoid and can be calculated using the formula for normal gravity on the ellipsoid. For our purposes some approximations are introduced and a particular version of $ \gamma $ for our normal heights system is obtained:
\begin{equation}
  \bar{\gamma} = \gamma - 0.1543 H \  mgal
\end{equation}

 
It is important to note that $ \bar{\gamma} $ returns a value in mgal units, where 100 gal is equivalent to $ 1m/s^{2} $ (or $ 1gal=1cm/s^{2} $). The formula for Vignal's Normal Height must be used carefully making sure all units correspond.

\subsection*{Results}
% Table generated by Excel2LaTeX from sheet 'Sheet1'
\begin{table}[htbp]
  \centering
  \caption{Normal Height Corrections for Points}
    \begin{tabular}{rc}
    \toprule
    \multicolumn{1}{c}{\textbf{Location Point}} & \textbf{Normal Height Correction (m)} \\
    \midrule
    Hermanus & 0.00016 \\
    Pretoria & 0.29259 \\
    Richards Bay & 0.00001 \\
    Thohoyandou & 0.06004 \\
    Ulundi & 0.05326 \\
    \bottomrule
    \end{tabular}%
  \label{tab:addlabel}%
\end{table}%


\renewcommand{\bibname}{References}
\bibliography{references}

\end{document}
