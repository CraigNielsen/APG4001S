
\section{Overview}
The objective of this paper is the determination of orthometric heights for five Trignet stations, given a GGMO2S Grace geopotential model to degree and order 160. The GRS80 ellipsoid characteristics will be used in the calculations.

In order to calculate the orthometric heights of the stations, the Geoidal height above the Ellipsoidal must be calculated as a function of gravity. The following steps breakdown this process.
\begin{enumerate}
	\item Find an expression for N as a function of the disturbing potential T
	\item Relate the disturbing potential T on the Geoid to the gravity anomoly $ \Delta g $
 This will allow an expression for N as a function of $ \Delta g $to be derived.
\end{enumerate}

\section{Find an expression for N as a function of the disturbing potential T }
This can be accomplished using Bruns formula.
\begin{equation} 
N = \dfrac{T_{p}}{\gamma_{0}}
\end{equation}

\section{Relate the disturbing potential T on the Geoid to the gravity anomoly}
$ \Delta g = (g_{P}- \gamma_{O})  $
and simplifying the Fundamental Gravimetric Equation to derive 
\begin{equation} 
\Delta g = -\dfrac{\delta T}{\delta n}-\dfrac{2}{a}.T
\end{equation}
using Laplace's Equation $ \Delta T \equiv 0 $
and the boundary condition above, solving for T.